% Use for better looking tables
%\usepackage{booktabs}

%
% Index creation
%
\usepackage{makeidx}
\makeindex

%
% The enumitem package improves itemize and enumerate lists
% just by using it. The most important reason to use it is
% so that exercises that begin with parts a. b. etc are formatted
% properly.
%
\usepackage{enumitem}
%\setlist[enumerate]{topsep=0pt}

%
% DRAFT watermark
%
\usepackage{draftwatermark}

%
% QR codes for URLs
%
\usepackage{qrcode}

%
% Define block structures
%

%
% We're not using amsthm because ntheorem is more flexible
%
\usepackage{ntheorem}

% For some reason, krantz.cls eats the whitespace before theorem environments.
% This fixes it manually.
\newlength{\pretheoremskiplen}
\setlength{\pretheoremskiplen}{10pt}

% Theorem and other mathy statements
\theoremstyle{plain}
\theorembodyfont{\itshape}
\theorempreskip{\pretheoremskiplen}
\theoremseparator{.}
\newtheorem{theorem}{Theorem}[chapter]
\newtheorem{lemma}{Lemma}[chapter]
\newtheorem{corollary}{Corollary}[chapter]
\newtheorem{proposition}{Proposition}[chapter]
\newtheorem{conjecture}{Conjecture}[chapter]

% Assumptions
\theoremstyle{plain}
\theorembodyfont{\itshape}
\theorempreskip{\pretheoremskiplen}

\newtheorem{assumptions}{Assumptions}[chapter]

% Definition
\theoremstyle{plain}
\theorembodyfont{\upshape}
\theorempreskip{\pretheoremskiplen}
\newtheorem{definition}{Definition}[chapter]

% Example
\theoremstyle{plain}
\theorembodyfont{\upshape}
\theorempreskip{\pretheoremskiplen}
\newtheorem{example}{Example}[chapter]

% Remark
\theoremstyle{nonumberplain}
\theorempreskip{\pretheoremskiplen}
\theorembodyfont{\upshape}
\newtheorem{remark}{Remark}

% Alert
\newenvironment{alert}%
   {%
      \vspace{4pt}\begin{leftbar}%
      \hspace{-1.5cm}\includegraphics[height=0.8cm]{images/alert.png}%
      \vspace{-1cm}%
   }%
  {\vspace{2pt}\end{leftbar}\vspace{-2pt}}

% Try it
\theoremstyle{nonumberbreak}
\theoremprework{\vspace{6pt}\begin{leftbar}\vspace{4pt}}
\theorempostwork{\vspace{2pt}\end{leftbar}\vspace{-2pt}}
\theorempreskip{0pt}
\theorembodyfont{\upshape}
\newtheorem{tryit}{Try It Yourself}
\theoremseparator{.}

% proof
\theoremstyle{nonumberplain}
\theorempreskip{\pretheoremskiplen}
\theorembodyfont{\upshape}
\theoremheaderfont{\itshape}
\theorempostwork{\vspace{-\pretheoremskiplen}\hfill\rule{1.4ex}{1.4ex}}
\newtheorem{proof}{Proof}

% Exercise
\theoremstyle{plain}
\theorempreskip{6pt}
\theorembodyfont{\upshape}
\newlength{\eSpace}
\settowidth{\eSpace}{ } % width of a space
\newtheorem{exercise}{\hspace{-\eSpace}}[chapter]

% Solutions
\usepackage{answers}
\Newassociation{solution}{soldisplay}{answerfile}
\begingroup   % hacky stuff to allow printing the percent character to the answerfile
\catcode`\%=12
\gdef\commentstring{% }  
\endgroup
\newcommand{\ExercisesBegin}{%
   \Opensolutionfile{answerfile}[answers/chapter\thechapter]%
   \Writetofile{answerfile}{\commentstring Auto-generated by answers package on \today}
   \Writetofile{answerfile}{\protect\SolutionsChapter{\thechapter}}%
}
\newcommand{\ExercisesEnd}{\Closesolutionfile{answerfile}}

% Alert (failed attempts)
%
% (non graphics triangle, unused)
%\newcommand\AlertTriangle{%
% \makebox[1.4em][c]{%
% \makebox[0pt][c]{\raisebox{.35em}{\large!}}%
% \makebox[0pt][c]{\Huge$\bigtriangleup$}}%
%}
%
% (failed attempt)
%\theoremstyle{nonumberbreak}
%\theoremprework{\vspace{\pretheoremskiplen}\begin{leftbar}}
%\theorempostwork{\end{leftbar}}
%\theoremseparator{}
%\theorempreskip{0pt}
%\theorembodyfont{\upshape}
%\newtheorem{alert}{\hspace{1ex}\AlertTriangle\hspace{1em}}
%


