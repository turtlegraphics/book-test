% Use for better looking tables
%\usepackage{booktabs}

%
% Index creation
%
\usepackage{makeidx}
\makeindex

%
% The enumitem package improves itemize and enumerate lists
% just by using it. The most important reason to use it is
% so that exercises that begin with parts a. b. etc are formatted
% properly.
%
\usepackage{enumitem}

%
% DRAFT watermark
%
\usepackage{draftwatermark}

%
% QR codes for URLs
%
\usepackage{qrcode}

%
% Define block structures
%

%
% We're not using amsthm because ntheorem is more flexible
%
\usepackage{ntheorem}

% For some reason, krantz.cls eats the whitespace before theorem environments.
% This fixes it manually.
\newlength{\pretheoremskiplen}
\setlength{\pretheoremskiplen}{10pt}

% Theorem and other mathy statements
\theoremstyle{plain}
\theorembodyfont{\itshape}
\theorempreskip{\pretheoremskiplen}
\theoremseparator{.}
\newtheorem{theorem}{Theorem}[chapter]
\newtheorem{lemma}{Lemma}[chapter]
\newtheorem{corollary}{Corollary}[chapter]
\newtheorem{proposition}{Proposition}[chapter]
\newtheorem{conjecture}{Conjecture}[chapter]

% Assumptions
\theoremstyle{plain}
\theorembodyfont{\itshape}
\theorempreskip{\pretheoremskiplen}

\newtheorem{assumptions}{Assumptions}[chapter]

% Definition
\theoremstyle{plain}
\theorembodyfont{\upshape}
\theorempreskip{\pretheoremskiplen}
\newtheorem{definition}{Definition}[chapter]

% Example
\theoremstyle{plain}
\theorembodyfont{\upshape}
\theorempreskip{\pretheoremskiplen}
\newtheorem{example}{Example}[chapter]

% Remark
\theoremstyle{nonumberplain}
\theorempreskip{\pretheoremskiplen}
\theorembodyfont{\upshape}
\newtheorem{remark}{Remark}

\newcommand\AlertTriangle{%
 \makebox[1.4em][c]{%
 \makebox[0pt][c]{\raisebox{.35em}{\large!}}%
 \makebox[0pt][c]{\Huge$\bigtriangleup$}}%
}
 
% Alert
%\theoremstyle{nonumberbreak}
%\theoremprework{\vspace{\pretheoremskiplen}\begin{leftbar}}
%\theorempostwork{\end{leftbar}}
%\theoremseparator{}
%\theorempreskip{0pt}
%\theorembodyfont{\upshape}
%\newtheorem{alert}{\hspace{1ex}\AlertTriangle\hspace{1em}}

% Alert
% (failed attempt)
%
% The marginnote package allows marginnotes with the command
% \marginnote
% that, unlike \marginpar, can happen from inside of boxes.
% Boxes like a framebox, which we need for alerts.
%
%\usepackage{marginnote}
%
% One of many failed attempts to put the alert logo in the left margin
\newenvironment{alert}%
   {%
      \vspace{4pt}\begin{leftbar}%
      \hspace{-1.5cm}\includegraphics[height=0.8cm]{images/alert.png}%
      \vspace{-1cm}%
   }%
  {\vspace{2pt}\end{leftbar}\vspace{-2pt}}

% Try it
\theoremstyle{nonumberbreak}
\theoremprework{\vspace{6pt}\begin{leftbar}\vspace{4pt}}
\theorempostwork{\vspace{2pt}\end{leftbar}\vspace{-2pt}}
\theorempreskip{0pt}
\theorembodyfont{\upshape}
\newtheorem{tryit}{Try It Yourself}
\theoremseparator{.}

% Exercise
\newlength{\eSpace}
\settowidth{\eSpace}{ } % width of a space
\theoremstyle{plain}
\theorempreskip{0pt}
\theorembodyfont{\upshape}
\newtheorem{exercise}{\hspace{-\eSpace}}[chapter]

% proof
\theoremstyle{nonumberplain}
\theorempreskip{\pretheoremskiplen}
\theorembodyfont{\upshape}
\theoremheaderfont{\itshape}
\theorempostwork{\vspace{-\pretheoremskiplen}\hfill\rule{1.4ex}{1.4ex}}
\newtheorem{proof}{Proof}

